%
% File acl2014.tex
%
% Contact: koller@ling.uni-potsdam.de, yusuke@nii.ac.jp
%%
%% Based on the style files for ACL-2013, which were, in turn,
%% Based on the style files for ACL-2012, which were, in turn,
%% based on the style files for ACL-2011, which were, in turn, 
%% based on the style files for ACL-2010, which were, in turn, 
%% based on the style files for ACL-IJCNLP-2009, which were, in turn,
%% based on the style files for EACL-2009 and IJCNLP-2008...

%% Based on the style files for EACL 2006 by 
%%e.agirre@ehu.es or Sergi.Balari@uab.es
%% and that of ACL 08 by Joakim Nivre and Noah Smith

\documentclass[11pt]{article}
\usepackage{acl2014}
\usepackage{times}
\usepackage{url}
\usepackage{latexsym}
\usepackage{amsmath}
%\setlength\titlebox{5cm}

% You can expand the titlebox if you need extra space
% to show all the authors. Please do not make the titlebox
% smaller than 5cm (the original size); we will check this
% in the camera-ready version and ask you to change it back.


\title{Instructions for ACL-2014 Proceedings}

\author{First Author \\
  Affiliation / Address line 1 \\
  Affiliation / Address line 2 \\
  Affiliation / Address line 3 \\
  {\tt email@domain} \\\And
  Second Author \\
  Affiliation / Address line 1 \\
  Affiliation / Address line 2 \\
  Affiliation / Address line 3 \\
  {\tt email@domain} \\}

\date{}

\begin{document}
\maketitle
\begin{abstract}
  This document contains the instructions for preparing a camera-ready
  manuscript for the proceedings of ACL-2014. The document itself
  conforms to its own specifications, and is therefore an example of
  what your manuscript should look like. These instructions should be
  used for both papers submitted for review and for final versions of
  accepted papers.  Authors are asked to conform to all the directions
  reported in this document.
\end{abstract}

%!TEX root = acl2014.tex
\section{Introduction}
Word representation is an important work in natrual laguage processing. It aims to represent a word as a vector that can well catch both the semantic relateness and syntactic information of words. And it's fundamention of many NLP tasks works. 
Recently, [Mikolov et al., 2013] proposed two efficient models, continuous bag-of-words model (CBOW) and Skip-Gram model, to learn word embeddings from a large unlabeled corpus. This method is now widely used in many NLP tasks. The CBOW model tries to combine the embeddings of context words to predict the target word; while SkipGram use the embedding of target word to predict its context words. 
However, they just use the information of word in the model. In Chinese, the charaters in each word also influence the meaning of words. And the components and radicals of each charaters can also contribute to the meaning of the chrater and the word. Therefore, in this paper, we take Chinese as a example and propose a new model for learning character,radical/components and word embeddings, with the use of the contexts ,characters and radicals information.
%!TEX root = acl2014.tex
\section{introduction}
Distributed word representation embeds a word into a continuous low dimentional vector and can better uncover both the semantic and syntactic information over traditional bag of words representations. It has been successfully applied in many downstream NLP tasks as input features, such as named entity recognition, sentiment analysis, and question answering. Among many embeding methods, CBOW and Skip-Gram model are very popular due to their simplicity and efficiency, making it feasible to learn good embeddings from a large scale training corpus. 

Despite the success and popularity of word embeddings, most of the existing methods treat each word as the mininum unit, which ignores the  morphological inormation of words. The representation of rare words may be poor despite that the training process of CBOW and Skip Gram typically subsample the frequent words. To address this issue, many recent works have investigated how to leverage morphological inormation to learn  better word embeddings. It has been proved efficive to improve word embedding quality, especially for morphologically rich languages.

Chinese is a kind of hieroglyphs and contains rich morphological information. The characters compsing a word can indicate the semantic meaning of the word and words sharing same character components always have similar meanings. Moreover, chinese characters can be broken into fine-grained components, which can be roughly divided into two types: semantic component and phonetic component. The semantic component represents the meaning of a character while the phonetic component represents the sound of a character. 
%For example, 氵is the semantic component of characters 河, 海 , 马  is the phonetic component of 妈 骂.

Leveraging these subword informations such as characters and character components can enrich chinese word embeddings with internal morphological semantics. Some methods have been proposed to incorporate these subword information for chinese word embeddings.  Sun Y et al. 2014 first introduced a radical-enhaced chinese character embedding model based on C\&W model and and  apply  it  on  Chinese  charac-ter similarity judgement and Chinese word segmentation. Yanran Li 2015 et al. developed two component-enhanced Chinese character embedding models and their bi-gram extensions based on CBOW and Skip Gram model.  Chen et al. proposed CWE model to joint learn chinsed word and character embedding and utilize the chinese characters to enrich chinese word embeddings. Jian Xu et al 2016 extends CWE work by exploiting the internal sematic similarity between a word and its characters  to combine word and character embedding  in a cross-lingual fashion. To combine the radical-character, character-word composition information, Rongchao Yin et al. 2016 propose multi-granularity embedding (MGE) model based on CWE model, which repsents the context as the combination of surrounding words, surrounding characters and the radical of target word to predict the target word. 

However, previous works only use character or radical of character to enrich a word embedding and don't make full use of fined grained components of characters. The component
%(部件构造)
 of chinese character is different from radicals, they are sometimes wrongly considered the same. Essentially, radicals are a specific set of characters that are used to index Chinese characters in dictionaries. Although many of them (not all) are also semantic components, each character has only one radical, which can not fully uncover the semantic and structure of a character. Besides, there are about 200 radicals while the number of components is over 10000.  Xinlei Shi et al. 2016 cut a character into fine-grained components according to wubi input method and get a these component embeddings by CBOW model. Then feed these component embeddings into deep neural networks and achives promising results in short-text categorization, chinese word segmentation, web search ranking tasks.

In this work, we present a model to jointly learn chinese word, character, sub character components embeddings. The learned chinese word embeddings can not only leverage the external context concurrence information but also incorporate rich internal subword structure and semantic information. Expreiments on both word similarity and word analogy tasks demonstrate the effectiveness of our model over previous works.
%!TEX root = acl2014.tex
\section{Joint Learning Model}
In this section, we introduces our joint leraning model, which combine word, character, sub character component information. Our model is based on CBOW model, we compare t effectiveness of two sub character features:
radical, component and two methods that combines word, character, sub character vectors in the context: JOIN1 and JOIN2. JOIN1 borrows the idea of  BEING (Fei Sun et al. 2016) and uses the sum of word vectors, the sum of character vectors, the sum of sub characters to predict te target word seperately and sum these three predict loss as the final loss function. JOIN2 follows the idea of CWE(Chen et al.) and represents the word  in the context as the composition of word embeddings , its characters embeddings and its sub character embeddings. Observing that in multi-granularity embedding (MGE) model(Yin et al. 2016), the radicals of the target word are used in the context to predict the target word, we also compare the performance of combining the surrounding words'sub character information and combining the target word's sub character information in our model.

Let $D = (w_1, w_2, \cdots, w_n)$ be the training corpus, $C = (c_1,c_2, \cdot, c_m)$ be the set of characters, $S= (s1,s_2,\cdots, s_n)$ be the set of sub characters, K be the window size. \\
\textbf{JOIN1} \quad For JOIN2 combining method, we aim to maximize the sum of three predictive loss for a target word $w_i$:
\begin{equation*}
L(w_i) = \log P(w_i | h_{i1}) + \log P(w_i | h_{i2}) + \log P(w_i | h_{i3})
\end{equation*}
where $h_{i1}, h_{i2}, h{i3}$ is the composition of context word embeddings, context character embeddings, context sub character embeddings resprectively. More precisely, we denote the surrounding words, characters, sub characters as the context , they can be represented as following:
\begin{equation*}
h_{i1} = \sum_{w_j \in context} w_j
\end{equation*}
\begin{equation*}
h_{i2} = \sum_{c_j \in context} c_j
\end{equation*}
\begin{equation*}
h_{i3} = \sum_{s_k \in context} s_k
\end{equation*}
The conditionaly probability is defined by the soft max function
\begin{equation*}
p(w_i |h_{i_k}) = \frac{\exp(h_i^T w_{i_k})}{\sum_{j = 1}^n \exp(h_{i_k}^T w_j)} \text{for k = 1, 2, 3}
\end{equation*}
The model aims to maximize the overall log likelihood :
\begin{equation*}
L(D) = \sum_{i = 1}^N L(w_i)
\end{equation*}
\textbf{JOIN2} \quad For JOIN2 combining method, the training objective is to maximize the following overall log likelihood:

\begin{equation*}
L(D) = \sum_{w_i \in D} \log p(w_i | h_i)
\end{equation*}
where $h_i$ is the vector composed by the embedding of context words, characters, and sub characters.
\begin{equation*}
h_i = \sum_{t = i - K, t \neq i}^{i+K} \frac{1}{2K}(w_t + \sum_{c_j \in w_t} \frac{1}{|w_t|} (c_j + \sum_{s_k \in c_j} \frac{1}{|c_j|}s_k))
\end{equation*}
$|w_i|$ represents the number of characters in word $w_i$ and $|c_j|$ represents the number of sub characters in charater $c_j$. \\
The conditional probability is defined as
\begin{equation*}
p(w_i |h_i) = \frac{\exp(h_i^T w_i)}{\sum_{j = 1}^n \exp(h_i^T w_j)}
\end{equation*}
%!TEX root = acl2014.tex
\section{Experiment Setup}

\textbf{Training Data} We use the Chinese Wikipedia as our training data source. In Chinese sentences, words are not separated by special symbols (as space in English sentences), so we firstly use THULAC\footnote{http://thulac.thunlp.org/} as the lexical analysis tool to separate words in sentences, then we can use this formated corpus in our experiments.

\textbf{Character Components} We crawled the component and radical information of Chinese characters from HTTPCN\footnote{http://tool.httpcn.com/}. This dataset contains 20879 characters, 13253 components and 218 radicals, of which 7744 characters have more than one components, and 214 characters are equal to their radicals.

\textbf{Testing Datasets} We found three datasets from work \textbf{???} which can be used to test Chinese word embeddings. The first one is for analogy testing, which consists of 1124 tuples of words and each tuple contains 4 words, coming from three different categories `Capital', `State' and `Family'. The words $w_i$ in each tuple $(w_1, w_2, w_3, w_4)$ in this dataset have the relationship that $w_2$ is to $w_1$ as $w_4$ is to $w_3$. The rest two are for similarity testing, both of which consist of 3-tuples $(w_1, w_2, s)$ where s is a real number representing the similarity between $w_1$ and $w_2$.

\textbf{Parameter Settings} We fix the word vector length to be 200, the window size to be 5, and the training iteration to be 1. Words with frequency less than 5 are ignored because they are too rare. The negative sampling size is set to be 10 and the subsampling parameter is set to be $10^{-3}$.

\textbf{Baseline} We use the CBOW model in work \textbf{???} as the baseline model. All the parameters are set to be the same as those in our model.

\textbf{Analogy Evaluation Metrics} In the analogy testing dataset, let $(w_1, w_2, w_3, w_4)$ be a tuple, then with a `good' word embedding $e_i$ of each word $w_i$, we can write down this form
\begin{align*}
	& e_2 - e_1 \approx e_4 - e_3 \\
	\Rightarrow & e_2 - e_1 + e_3 \approx e_4 \\
	\Rightarrow & (e_2 - e_1 + e_3)\cdot e^{(\ell)} \approx e_4\cdot e^{(\ell)}.
\end{align*}
This form shows $(e_2 - e_1 + e_3)\cdot e^{(\ell)}$ is an approximation of $e_4\cdot e^{(\ell)}$, which is the similarity between $e_4$ and $e^{(\ell)}$. If we calculate $(e_2 - e_1 + e_3)\cdot e^{(\ell)}$ for each $w^{(\ell)}$, we are expected to get the maximum result when $w^{(\ell)}=w_4$. In our experiments, with a word embedding and a tuple, we pick the word with maximum approximate similarity (except for the first three words in this tuple) as the prediction of the fourth word, then the prediction precision over all tuples are used as the measurement of the quality of this embedding.

\textbf{Similarity Evaluation Metrics} In this part, we evaluate the quality of an embedding by a ranking-correlation method. For all 3-tuples $(w_1, w_2, s)$ in a similarity testing dataset, we can calculate the similarity $s'$ between $w_1$ and $w_2$ with an embedding, then we calculate the Spearsman Correlation between all the $s$ and $s'$ as the quality of this embedding.

%!root = acl2014.tex
Word Relatedness Computation
In this task, we use two different datasets, wordsim-240 and wordsim-296 to evaluate the semantic relateness of word emmbeddings given by each model. The correlations between results of models and human judgements are reported as the model performance. 
In wordsim-240, there are 240 pairs of Chinese words and human-labeled relatedness scores, of which the 233 word pairs have appeared in the learning corpus. In wordsim-296, the words in 280 word pairs have appeared in the learning corpus and the left 16 pairs have new words.
We compute the Spearman correlation ρ between relatedness scores from a model and the human judgements for comparison. The evaluation results of our model and baseline methods on wordsim-240 and wordsim-296 are shown in Table 2
Analogical Reasoning

\section*{Acknowledgments}

The acknowledgments should go immediately before the references.  Do
not number the acknowledgments section. Do not include this section
when submitting your paper for review.

% include your own bib file like this:
%\bibliographystyle{acl}
%\bibliography{acl2014}

\section{bibliography}

[1] Tomas Mikolov, Kai Chen, Greg Corrado, and Jeffrey Dean. Efficient Estimation of Word Representations in Vector Space. In Proceedings of Workshop at ICLR, 2013\\
[2] Xinlei Shi, et al. Radical embedding: Delving deeper to chinese radicals. ACL 2015\\
[3] Sun Y, Lin L, Yang N, et al. Radical-enhanced chinese character embedding[C]//International Conference on Neural Information Processing. Springer International Publishing, 2014: 279-286.\\
[4] Yanran Li, Wenjie Li, Fei Sun, and Sujian Li. 2015. Component-enhanced chinese character embeddings. In Proceedings of the 2015 Conference on Empirical Methods in Natural Language Processing.\\
[5] Xinxiong Chen, Lei Xu, Zhiyuan Liu, Maosong Sun, Huanbo Luan. Joint Learning of Character and Word Embeddings. The 25th International Joint Conference on Artificial Intelligence (IJCAI 2015).\\
[6] Xu J, Liu J, Zhang L, et al.  Improve Chinese Word Embeddings by Exploiting Internal Structure[C]//Proceedings of NAACL-HLT. 2016: 1041-1050.\\
[7] Rongchao Yin, et al.  Multi-Granularity Chinese Word Embedding. ACL 2016\\
[8] Sun F, Guo J, Lan Y, et al. Inside Out: Two Jointly Predictive Models for Word Representations and Phrase Representations[C]//Thirtieth AAAI Conference on Artificial Intelligence. 2016.\\

\end{document}
